\chapter{A (too short) introduction to R}

This is a very brief introduction to \R (which can be downloaded from the website \texttt{www.r-project.org}) and its capabilities. It will be extremely focussed on the characteristics that are instrumental to do health economic evaluations using a combination of \R, \bugs and some useful packages (such as \bcea). Thus it is by no means exhaustive!

When you open the \R terminal, you are presented with the possibility of typing commands. You may want to open a text editor (e.g. the simple one built into the R Windows interface) in which you can type directly these commands, and save them to a script for future use. Another possibility is to use \R from within a more sophisticated ``integrated development environment'', such as \texttt{RStudio} (\texttt{www.rstudio.com}), which has many more features than the basic Windows interface.

In any case, \R is a very powerful tool; more importantly, it is free and you can find a wealth of documentation on the internet. \R has a set of built-in commands, which you can use for basic operations. However, there are also many add-on packages containing sets of functions designed to perform specific statistical tasks. These packages can be installed to \textit{your} \R by typing the command
\begin{lstlisting}
> install.packages("package_name") 
\end{lstlisting}
(assuming you have an internet connection and noticing that the symbol \texttt{>} indicates the beginning of a line of code in \texttt{R}). This command only needs to be executed one time.  Once a package is installed in your local \texttt{library} (a collection of packages) you can make it available to the current \R session by typing the command
\begin{lstlisting}
> library(package_name)
\end{lstlisting}

For these practicals, you will need to install and load the packages:
\begin{itemize}
\item \bcea, which can be used to post-process the results of a (Bayesian) model to perform a health economic evaluation. 
\item \texttt{R2OpenBUGS}, which can be used to interface \R and \bugs.
\end{itemize} 
You do this by typing in your \R terminal the commands
\begin{lstlisting}
> install.packages("BCEA")
> install.packages("R2OpenBUGS")
> library(BCEA)
> library(R2OpenBUGS)
\end{lstlisting}
Both \bcea and \texttt{R2OpenBUGS} will automatically load other packages that they \textit{depend on} --- this means that in order to work, they need to access functions that are part of other packages.

If you wish so, you can use \texttt{JAGS} in the practicals. To this end (and assuming you have actually installed the current version of \texttt{JAGS} to your computer), you will need to also install the package \texttt{R2jags}, which you can do by typing in your \R terminal
\begin{lstlisting}
> install.packages("R2jags")
\end{lstlisting}
Notice that if you decide to use \jags instead of \bugs, you will need to slightly modify some of the commands --- we describe this in more details later in this manual.

Once a package is loaded to your \R workspace, you can type the command \texttt{help(package\_name)}, which will open a window displaying a description of the package. For example \texttt{help(BCEA)} provides some basic information (including details of the current version). You can use the command \texttt{help} also on specific functions within the package, e.g.\ typing \texttt{help(bcea)} describes in detail how to use the \texttt{bcea} function (notice that in this case the package name is typeset in uppercase, while the function is lowercase!).

The very basic commands that are required to do a typical \R session working with \bugs and \bcea will be given and described later or in the scripts that we refer to in the practicals.