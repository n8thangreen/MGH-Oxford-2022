\documentclass[a4paper,twoside,openany]{../svmonoBUGS}\usepackage[]{graphicx}\usepackage[]{color}
% maxwidth is the original width if it is less than linewidth
% otherwise use linewidth (to make sure the graphics do not exceed the margin)
\makeatletter
\def\maxwidth{ %
  \ifdim\Gin@nat@width>\linewidth
    \linewidth
  \else
    \Gin@nat@width
  \fi
}
\makeatother

\definecolor{fgcolor}{rgb}{0.345, 0.345, 0.345}
\newcommand{\hlnum}[1]{\textcolor[rgb]{0.686,0.059,0.569}{#1}}%
\newcommand{\hlstr}[1]{\textcolor[rgb]{0.192,0.494,0.8}{#1}}%
\newcommand{\hlcom}[1]{\textcolor[rgb]{0.678,0.584,0.686}{\textit{#1}}}%
\newcommand{\hlopt}[1]{\textcolor[rgb]{0,0,0}{#1}}%
\newcommand{\hlstd}[1]{\textcolor[rgb]{0.345,0.345,0.345}{#1}}%
\newcommand{\hlkwa}[1]{\textcolor[rgb]{0.161,0.373,0.58}{\textbf{#1}}}%
\newcommand{\hlkwb}[1]{\textcolor[rgb]{0.69,0.353,0.396}{#1}}%
\newcommand{\hlkwc}[1]{\textcolor[rgb]{0.333,0.667,0.333}{#1}}%
\newcommand{\hlkwd}[1]{\textcolor[rgb]{0.737,0.353,0.396}{\textbf{#1}}}%
\let\hlipl\hlkwb

\usepackage{framed}
\makeatletter
\newenvironment{kframe}{%
 \def\at@end@of@kframe{}%
 \ifinner\ifhmode%
  \def\at@end@of@kframe{\end{minipage}}%
  \begin{minipage}{\columnwidth}%
 \fi\fi%
 \def\FrameCommand##1{\hskip\@totalleftmargin \hskip-\fboxsep
 \colorbox{shadecolor}{##1}\hskip-\fboxsep
     % There is no \\@totalrightmargin, so:
     \hskip-\linewidth \hskip-\@totalleftmargin \hskip\columnwidth}%
 \MakeFramed {\advance\hsize-\width
   \@totalleftmargin\z@ \linewidth\hsize
   \@setminipage}}%
 {\par\unskip\endMakeFramed%
 \at@end@of@kframe}
\makeatother

\definecolor{shadecolor}{rgb}{.97, .97, .97}
\definecolor{messagecolor}{rgb}{0, 0, 0}
\definecolor{warningcolor}{rgb}{1, 0, 1}
\definecolor{errorcolor}{rgb}{1, 0, 0}
\newenvironment{knitrout}{}{} % an empty environment to be redefined in TeX

\usepackage{alltt}
\title{Bayesian methods in health economics \\ STAT3021/STATM021/STATG021}
\author{Practicals}
\date{January -- March 2018}

\usepackage{xspace,graphicx,bm,amssymb,amsmath,url,graphics,inconsolata}
\usepackage[text={15.2cm,22cm},centering]{geometry}
\usepackage{inconsolata}
\usepackage[T1]{fontenc}
\usepackage[normalem]{ulem}
\usepackage{charter}
\usepackage{tikz,verbatim,graphicx,color}
%%%\usetikzlibrary{shapes,arrows,arrows.meta,decorations.pathreplacing,shapes.geometric}


\newcommand{\winbugs}{{\texttt{WinBUGS}}\xspace}
\newcommand{\openbugs}{{\texttt{OpenBUGS}}\xspace}
\newcommand{\jags}{{\texttt{JAGS}}\xspace}
\newcommand{\bugs}{{\texttt{BUGS}}\xspace}
\newcommand{\R}{{\texttt{R}}\xspace}
\newcommand{\bcea}{{\texttt{BCEA}}\xspace}

\usepackage{listings}
\lstset{basicstyle=\ttfamily\fontsize{9}{11}\selectfont,breaklines=true,tabsize=2,
keywords={},linewidth=1\textwidth,backgroundcolor=\color{black!5}} 

\renewcommand{\chaptername}{Practical}

\graphicspath{{/home/gianluca/Dropbox/EcSan/ShortCourses/BSU-UCL/LaTeX/}{/home/gianluca/Dropbox/UCL/3021/Lectures/figs/}}
\IfFileExists{upquote.sty}{\usepackage{upquote}}{}
\begin{document}
\addtocounter{chapter}{7}
\chapter{Decision trees --- SOLUTIONS}

%
\section{Running and analysing results}

Run the file "practical.R".
You can do this using the R command \texttt{source("practical.R")}

The required R code is 

\begin{verbatim}
# Net benefit
(net.benefit <- lambda.target * effects - costs)

# Incremental results relative to no treatment
(incremental.costs <- costs - costs[1])
(incremental.effects <- effects - effects[1])
(incremental.net.benefit <-
    lambda.target * incremental.effects - incremental.costs)

# Incremental cost effectiveness ratios relative to no treatment
(icer <- incremental.costs / incremental.effects)
\end{verbatim}

Run the file "practical\_probs.R".
To ensure the input parameters make sense, look at the means of matrices using the \texttt{colMeans()} function
and the means of vectors using the \texttt{mean()} function

For matrices, CBT has lowest probability of relapse
\begin{verbatim}
colMeans(p.rel)
\end{verbatim}

Antidepressant has the highest probability of recovery
\begin{verbatim}
colMeans(p.rec)
\end{verbatim}

Mean cost of treatment (these are actually constant!)
\begin{verbatim}
colMeans(c.treat)
\end{verbatim}

For vectors, Cost of no recovery
\begin{verbatim}
mean(c.norec)
\end{verbatim}

Cost of relapse
\begin{verbatim}
mean(c.rel)
\end{verbatim}

Cost of recovery
\begin{verbatim}
mean(c.rec)
\end{verbatim}

The cost of no recovery is highest, as patients incur higher hospitalization and therapy costs.
QALYs of no recovery
\begin{verbatim}
mean(q.norec)
\end{verbatim}

QALYs of relapse
\begin{verbatim}
mean(q.rel)
\end{verbatim}

QALYs of recovery
\begin{verbatim}
mean(q.rec)
\end{verbatim}

The QALY for recovery is highest, as patients don't suffer a relapse and spend most of remaining 30 years if good mental health.

\subsection{Summary of cost and effects}

Mean QALYs of each treatment over 30 years.
Very similar but antidepressants and CBT are higher than no treatment.
\begin{verbatim}
colMeans(effects)
\end{verbatim}

Mean costs
CBT is the most expensive while no treatment and antidepressant are similar.
\begin{verbatim}
colMeans(costs)
\end{verbatim}

Calculate the net benefit at a willingess-to-pay of £20,000.
\begin{verbatim}
net.benefit <- 20000*effects-costs
\end{verbatim}

Net benefit is highest for antidepressant, but all options are very close.
\begin{verbatim}
colMeans(net.benefit)
\end{verbatim}

%
\section{Using BCEA}

If BCEA isn't yet installed you'll have to install BCEA first
\begin{verbatim}
install.packages("BCEA")
\end{verbatim}

Load the package
\begin{verbatim}
library(BCEA)
\end{verbatim}

\begin{itemize}
\item Create a \texttt{bcea} object for the depression decision tree
\texttt{e} are the effects, \texttt{c} are the costs
Set the \texttt{ref} reference treatment to be 1 (no treatment)
Set the interventions, which is the names of the treatments, to \texttt{t.names}
\begin{verbatim}
depression.bcea <- bcea(e=effects,c=costs,ref=1,interventions=t.names)
\end{verbatim}

\item
The \texttt{summary()} gives comparisons of CBT and antidepressants to no treatment. 
wtp is the willingness-to-pay threshold. The detafult to 25000 so set this to 20000
The "EIB" is expected incremental benefit at the \texttt{wtp=20000}, the "CEAC" is the
probability that the reference of "no treatment" has highest net benefit (most cost-effective), 
and the ICER is the incremental cost-effectiveness ratio. 

We can see that EIB is negative for no treatment relative to both CBT and antidepresants, so the latter two are favoured
The CEAC is less than 0.5 so there is a higher probability that CBT and antidepressant are most cost-effective
The ICER is less than £20,000 for both CBT and antidepressant so both should be considered cost-effective
\begin{verbatim}
summary(depression.bcea, wtp=20000) 
\end{verbatim}

\item
Now comparing CBT to antidepressant
Use the \texttt{bcea()} function but with \texttt{ref=2}, so that CBT is the reference treatment.
All other inputs remain the same
\begin{verbatim}
depression.refCBT.bcea <- bcea(e=effects,c=costs,ref=2,interventions=t.names)
\end{verbatim}

The final matrix of EIB and CEAC are of interest.
The first row are comparisons with no treatment which suggest CBT is more cost-effective as EIB is positive,
CEAC>0.50. The ICER isn't useful for interpretation.
The second row tells us that CBT has lower incremental net benefits (EIB is negative) and that the probability
that CBT is more cost-effective than antidepressants is less than 50\% (CEAC<0.50)
\begin{verbatim}
summary(depression.refCBT.bcea, wtp=20000)
\end{verbatim}

\item Multiple treatment comparison
For multiple treatment comparison pass the output of\texttt{ bcea()} to \texttt{multi.ce()}
This may take a moment
\begin{verbatim}
depression.multi.ce <- multi.ce(depression.bcea)
\end{verbatim}

Now generate the probability that each treatment has the highest net benefit at
a range of willingness-to-pay thresholds.
These are the cost-effectiveness acceptability curves (CEAC).
At £20,000, antidepressants have the highest probability of having highest net benefit (being most cost-effective).
\begin{verbatim}
ceac.plot(depression.multi.ce,
          pos = c(1, 0.5))
\end{verbatim}
\includegraphics{multi\_ceac}
\end{itemize}

\end{document}
